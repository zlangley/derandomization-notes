\newcommand{\fH}{\mathbb{H}}

\chapter{Doubly-Efficient Proof Systems}
\label{lec:09}

\section{Logspace Uniform Circuits}

\begin{definition}\label{def:logspaceuniform}
  A circuit family $\{C_n\}_{n \geq 1}$ is called \emph{logspace-uniform} if
  there exists a TM which on input $1^n$ outputs a description of the circuit
  $C_n$ of size $s(n)$ in space $O(\log s(n))$.
\end{definition}
\commenta{Zach}{I am confused about the size $s(n)$ is the definition above. So
the TM can run in exponential time as long as the circuit output is doubly
exponential? I feel like it should use space $\log{n}$ not $\log{s(n)}$?}

Note that there are many notions of uniformity. Until now, we have been
looking at $\P$-uniform circuit classes, where the circuits can be generated by
a Turing machine in polynomial time. The above definition is more restricted,
as the Turing machines generating the circuits may only use
logarithmic space.

\begin{definition}
  Let $p : \N \to \N$ and let $v : \N \to \N$. The class $\DEIP[p, v]$ is the
  set of all languages $L$ such that there is an interactive protocol that
  accepts $L$ with completeness and soundness at least $\nicefrac{2}{3}$, and on inputs
  of size $n$, the prover runs in time $p(n)$ and the verifier runs in time
  $v(n)$.
\end{definition}

Observe that if $p$ and $v$ are polynomials, then $\DEIP[p, q] = \BPP$. The
following theorem relates the class $\DEIP$ to logspace uniform circuits. In
particular, it implies that $\L$-uniform $\NC$ is contained in $\DEIP[\poly(n),
\widetilde{O}(n)]$.

\begin{theorem}\label{thm:main}
  Any language $L$ that can be computed by a family of $O(\log s(n))$-space
  uniform boolean circuits of size $s(n)$ and depth $d(n)$ has an interactive
  proof such that:
	\begin{enumerate}
		\item An honest prover runs in time $O(\poly\,{s(n)})$;
		\item The verifier runs in time $O(n \cdot \poly(d(n), \log s(n)))$ and space $O(\log s(n))$;
		\item The protocol has completeness 1 and soundness 1/2;
    \item The protocol uses $O(d(n)\polylog\,{s(n)})$ bits of communication with
      public randomness.
  \end{enumerate}
\end{theorem}

\section{Circuit Specification and Arithmetization}

The prover and the verifier are given a boolean circuit $C$ of size $s$ and
depth $d$ and an input $x$. The prover wants to convince a space-bounded
verifier that $C(x) = 1$.

We assume the circuit $C$ has the following form:
\begin{enumerate}
  \item The fan-in of each gate in the circuit is 2. (Any circuit can be
    transformed into one with fan-in 2 by increasing the depth of the circuit
    by a factor of $O(\log s)$.)
  \item The circuit is layered: a gate is in layer $i$ if it is at distance $i$
    from an output gate, and gates in layer $i$ only recieve their inputs from
    layer $i + 1$. (We can always add intermediary forwarding gates if this is
    not the case.)
  \item The circuit has exactly $s$ gates at each layer. (We can add dummy
    gates that do not feed into any other gates.)
  \item Each gate in the circuit is a NAND gate. (NAND gates are universal and
    any other $\wedge$ or $\vee$ gates can be replaced by a constant number of
    NAND gates.)
\end{enumerate}

We number the gates in each layer in the circuit from $\{0,1,\ldots, s-1\}$.
Let $g_{i,j}$ denote the gate in layer $i$ at position $j$.

\begin{definition}\label{def:psis}
  We define the \emph{circuit gate functions} $\psi_i:\{0,1,\ldots, s-1\}^3 \to
  \{0,1\}$ for $i \in \{0, 1, \ldots, d-1\}$ by
	\[
    \psi_i(a,b,c) = \begin{cases}
      1 &\text{if $g_{i+1,b}$ and $g_{i+1,c}$ feed into $g_{i,a}$}\\
      0 &\text{otherwise}
    \end{cases}
	\]
  The $d$ functions defined above capture the incidence structure of $C$ and
  thus determine $C$.
\end{definition}

Let $\F$ be a field and $\fH$ be a subset of $\F$.

\commenta{Zach}{In the lemma below we need that we can generate a circuit for
$W$ too, right?}

\begin{lemma}\label{lem:lde}
  Given a function $W:\fH^k \to \fH$ on $k$ variables, we can construct another
  function $\widetilde{W}:\F^k \to \F$ which agrees with $W$ on all inputs $x
  \in \fH^k$ and has degree at most $|\fH|-1$ in each variable. A circuit for
  $\tilde{W}$ can both be generated and evaluated in $O(\poly(|\F|^k))$ time
  and $O(\log k + \log |\F|)$ space. This function $\tilde{W}$ is called the
  low degree extension of $W$.
\end{lemma}

\begin{proof}
  First we define the function $\delta : \F^k \times \F^k \to \F$ by
  \begin{equation}
    \delta(x,y) = \prod_{i=1}^k \sum_{a \in \fH} \prod_{b \in \fH\setminus\{a\}} \frac{(x_i-b)(y_i-b)}{(a-b)^2}
  \end{equation}

  Note that for $x, y \in \fH$, we have $\delta(x, y) = 1$ if and only if $x =
  y$. For other inputs, the output of $\delta$ is just some element in $\F$.

  By construction, the degree of $\delta$ is at most $|\fH|-1$ in each variable
  and $\delta$ has an arithmetic circuit of size $O(\poly(|\F|, k))$. (The
  circuit contains $k-1$ product gates and a smaller circuit of size $|\fH|^2$
  as input to these product gates.) This circuit can both be generated and
  evaluated in $O(\poly(|\F|, k))$ time and $O(\log k + \log |\F|)$ space.

  Using $\delta$ we define $\widetilde{W}$ as
	\begin{equation*}
    \widetilde{W}(x) = \sum_{a \in \fH^k} \delta(a,x) W(a).
	\end{equation*}
  It's easy to see that the function $\widetilde{W}$ has an arithmetic circuit
  of size $O(|\F|^k \cdot \poly(k , |\F|))$ by the above construction. This
  arithmetic circuit can be evaluated and generated in time $O(|\F|^k \cdot
  \poly(k , |\F|))$ and space $O(m \log (|\F|))$.
\end{proof}

Henceforth, for any function $f: \fH \to \fH$, we use $\tilde{f}:\F \to \F$ to
denote the low degree extension of $f$ to field $\F$. We pick two fields $\fH$
and $\F$ and an integer $m$ with $\{0,1\} \subset \fH \subset \F$ such that
\begin{equation}
  \log s \leq |\fH| \leq \poly(\log s),
\end{equation}
\begin{equation}
  |\F| \leq \poly(|\fH|),
\end{equation}
and
\begin{equation}
  s \leq |\fH|^m \leq \poly(s).
\end{equation}

The prover needs to operate over $\F^m$ and the verifier works with the fields
$\F$ and $\fH$. This helps us understand why the parameters have been defined
as shown above.

We need $|\F| \geq \poly(\log s, d)$ for error correction to get good
soundness. If $|\fH| = o(\log s)$, then $\F^m$ would be $\omega(\poly(s))$, as
the value of $m$ would be too large. This would make the prover require super
polynomial space and time, but we want the prover to be efficient.

Now we move on to arithmetizing the circuit $C$. Throughout this section, we
fix the input $x$, as this is known to both the prover and the verifier. The
prover is interested in proving $C(x)=1$ for a fixed $x$.

Each layer $i$ of $C$ can be viewed as a value $\ell_i \in \{0,1\}^{s}$, where
the value at position $j$ is the output of the gate at position $j$ in layer
$i$. For example, the input layer $\ell_d$ is $x0^{n-s}$ (we can assume that
the dummy gates have output 0). The output layer $\ell_0$ is $C(x)0^{s-1}$.

Function $L_i:\{0,1, \ldots, s-1\} \rightarrow \{0,1\}$ for $i \in [d]$ is used
to fix the output on layer $i$. $L_i(j)$ is defined as the output on gate $j$
in layer $i$ of $C$.

If we assign a mapping from $\{0,1, \ldots, s-1\}$ to $\fH^m$ (size of $\fH^m$
is greater than $s$) which is easy to compute, then $L_i$ can be seen as a
function from $\fH^m$ to $\{0,1\}$. The mapping can be any one-one mapping.

These $d$ functions $L_i$s fix the output on $x$ on all the gates of the
circuit, and our $d$ functions $\psi_i$s from \cref{def:psis} describe the
circuit $C$. We define $\tilde{L_i}: \F^m \rightarrow \F, \tilde{\psi_i}:
\F^{3m} \rightarrow \F$ for $i\in \{0,1,\ldots, d\}$ as their low degree
extensions from \cref{lem:lde}, and work with these functions.

\section{Barebones Sumcheck Protocol}

In this section we prove \cref{thm:main} when the verifier has access to
oracles of low degree extensions of circuit specification functions from
\cref{def:psis}. In the next section, we show how the verifer can compute
$\tilde{\psi_i}$s by using the logspace uniformity (from
\cref{def:logspaceuniform}) of $C$.

The prover wants to prove to the verifier that $\tilde{L_0}(1^m) = 1$ if $1^m$
is the value in $\fH^m$ mapped to the output gate. The prover can reduce this
task to proving that $\tilde{L_1}(z_1) = r_1$ for $z_1, \in \F^m$, $r_1 \in
\F$. More generally, the prover can reduce the task of proving
$\tilde{L_i}(z_i) = r_i$ in some layer $i$ to the verifier to the task of
proving that $\tilde{L_{i+1}}(z_{i+1}) = r_{i+1}$ in the next layer $i+1$ by
running a sumcheck protocol. After $d$ different sumcheck protocols, the
verifier has to check whether $\tilde{L_d}(z_d) = r_d$ in the input layer,
which it can verify without the aid of the prover in $O(n\cdot \poly(d, \log
s))$ space.

\section{Highly Uniform Circuit Oracles}