\begin{lecture}{1}{The Class \pr{BPP}}{January 23, 2020}
\label{lec:01}

\subsection*{The Upshot}

\begin{enumerate}
  \item Randomness is necessary in many computational settings---is it
    necessary for polynomial-time computation? This is a fundamental question
    we will investigate in this course.
  \item Promise problems are a natural generalization of languages, and the
    theory of probabilistic algorithms for promise problems is richer than
    for that of languages.
  \item The class \pr{BPP} has a complete problem, The Circuit Acceptance
    Probability Problem (CAPP), which asks whether an input circuit has
    acceptance probability at least 2/3.
  \item The class \pr{BPP} admits a time-hierarchy theorem.
  \item Probabilistic search problems can be reduced to probabilistic decision
    problems, meaning the theory we will develop around the class of decision
    problems \pr{BPP} will apply to the search analog as well.
\end{enumerate}

\subsection{Equivalent Definitions of \pr{BPP}}

A \emph{promise problem} is a pair $\Pi = (\Pi_Y, \Pi_N) \subseteq \{0, 1\}^*
\times \{0, 1\}^*$ such that $\Pi_Y \cap \Pi_N = \emptyset$. Promise problems
generalize languages in the case that $\Pi_Y \cup \Pi_N = \Pi$.

The complexity class \pr{BPP} is the set of all promise problems $\Pi$ such
that there exists a probabilistic polynomial-time Turing machine $M$ deciding
$\Pi$ with probability $2/3$. By a classic amplification technique, the
correctness probability $2/3$ can be replaced with any other constant and we
obtain the same set of problems.

\subsection{The Circuit Acceptance Probability Problem (CAPP)}

\subsection{A Time-Hierarchy Theorem}

\subsection{Reducing Search Problems to Decision Problems}

\subsection{Two-Sided Error vs.\ One-Sided Error}

\end{lecture}
