\documentclass[12pt]{article}

\input{preamble.tex}

\usepackage{mathtools}

\def\eps{\varepsilon}
\def\Exp{\mathbb{E}}
\def\N{\mathbb{N}}
\def\R{\mathbb{R}}

\def\sharpp{\textsf{\#P}\xspace}

\DeclareMathOperator{\Bern}{Bern}
\DeclarePairedDelimiter\brac{[}{]}
\DeclarePairedDelimiter\card{\lvert}{\rvert}
\DeclarePairedDelimiter\set{\{}{\}}
\DeclarePairedDelimiter\floor{\lfloor}{\rfloor}
\DeclarePairedDelimiter\paren{(}{)}

\newcommand{\comment}[1]{\textcolor{red}{[\textbf{#1}]}}

\newlang{\CAPP}{CAPP}
\newlang{\PERM}{PERM}

\renewclass{\promiseBPP}{prBPP}
\newclass{\prBPP}{prBPP}
\newclass{\prBPTIME}{prBPTIME}
\newclass{\prRP}{prRP}
\newclass{\prcoRP}{prcoRP}
\newclass{\prP}{prP}

\renewcommand{\Pr}{\mathbb{P}}
\addbibresource{main.bib}

\title{\bf Derandomization and Its Connections Throughout Complexity Theory}
\author{}

\begin{document}
  \allsectionsfont{\sffamily}

  \newcommand{\pr}[1]{\textsf{pr#1}}
  \newcommand{\dtime}[1]{\DTIME\brac*[#1]}

  \maketitle

  % Table of contents with "Contents" line.
  \makeatletter
  \@starttoc{toc}
  \makeatother

  \begin{lecture}{1}{The Class prBPP}{January 23, 2020}
\label{lec:01}

\subsection*{The Upshot}

\begin{enumerate}
  \item Randomness is necessary in many computational settings. Is randomness
    necessary for polynomial-time computation? There are good reasons to expect
    it is not! This is a fundamental question we will investigate in this
    course.
  \item Promise problems are a natural generalization of languages, and the
    theory of probabilistic algorithms for promise problems is richer than
    for that of languages.
  \item The class $\prBPP$ has a complete problem ($\CAPP$) under deterministic
    polynomial-time reductions. Hence to prove $\prBPP = \prP$ it suffices
    to give a deterministic polynomial-time algorithm for $\CAPP$.
  \item The class $\prBPP$ admits a time-hierarchy theorem.
  \item Probabilistic search problems can be reduced to probabilistic decision
    problems, meaning the theory we will develop around the class of decision
    problems $\prBPP$ will apply to the search analog as well.
\end{enumerate}

\subsection{Promise Problems and prBPP}

A \emph{promise problem} is a pair $\Pi = (\Pi_Y, \Pi_N) \subseteq \{0, 1\}^*
\times \{0, 1\}^*$ such that $\Pi_Y \cap \Pi_N = \emptyset$. The set $\Pi_Y$
captures all of the ``yes'' instances, and the set $\Pi_N$ captures all of the
``no'' instances. The set $\Pi_Y \cup \Pi_N$ is called the \emph{promise}. We
say that a Turing machine $M$ \emph{solves} a promise problem $\Pi$ if on
input $x \in \Pi_Y$ it accepts and on input $x \in \Pi_N$ it rejects. Crucially,
for inputs not in the promise, the machine can behave arbitrarily.

Promise problems generalize decision problems (languages) in the case that
$\Pi_Y \cup \Pi_N = \{0,1\}^*$. Complexity classes based on decision problems
have obvious analogs based on promise problems. For example, $\prP$ is the
class of all promise problems solvable by a deterministic polynomial-time
Turing machine. Similarly, the class $\prBPP$ is the class of promise problems
solvable by a probabilistic polynomial-time Turing machine with error
probability at most 1/3 for all instances. The class $\prBPP$ will be one of
our main objects of study, so its definition is worth repeating carefully.

\begin{definition}[$\prBPP$]
  A promise problem $\Pi = (\Pi_Y, \Pi_N)$ is in $\prBPP$ if and only if there
  exists a polynomial $p(n)$ and a probabilistic Turing machine $M$ such that
  \begin{enumerate}
    \item if $x \in \Pi_Y$, then $M$ accepts input $x$ with probability greater
      than $2/3$ after $p(n)$ steps, and
    \item if $x \in \Pi_N$, then $M$ rejects input $x$ with probability greater
      than $2/3$ after $p(n)$ steps.
  \end{enumerate}
\end{definition}

The constant $2/3$ in the definition of $\prBPP$ is not important; it
can be replaced by any other constant greater than 1/2, which we will discuss
in more detail in the next section.%

\begin{proposition}
  If $\prBPP = \prP$, then $\BPP = \P$.
\end{proposition}

\begin{proof}
  If $\Pi \in \BPP$, then trivially $\Pi \in \prBPP$, and therefore by
  assumption, $\Pi \in \prP$. As $\Pi$ is a decision problem, it follows
  that $\Pi \in \P$.
\end{proof}

What do we know about the placement $\BPP$ with respect to other complexity
classes? An easy inclusion is $\BPP \subseteq \PSPACE$.

\begin{proposition}
  $\BPP \subseteq \PSPACE$.
\end{proposition}
\begin{proof}
  In one sentence: Simulate the $\BPP$ algorithm with all possible random
  strings.

  In more detail, let $L \in \BPP$ and let $M$ be a probabilistic Turing
  machine that solves $L$ in polynomial time $p(n)$. We can design a
  deterministic machine $M'$ that solves $L$ as follows. On input $x$, $M'$
  simulates $M$ on input $(x, r)$ for every $r \in \set{0,1}^{p(n)}$ and counts
  the number of times $M$ accepts. If $M$ accepts for more than $\frac{2}{3}
  \cdot 2^{p(n)}$ choices of $r$, then $M'$ accepts. Otherwise, $M'$ rejects.

  Clearly $M'$ accepts $L$ if and only if $M$ accepts $L$. The space
  requirements of $M'$ are the space requirements of $M$ plus the $p(n)$ bits
  for $r$ plus the $p(n)$ bits for the counter, and hence $M$ decides $L$ in
  polynomial space.
\end{proof}

Of course we cannot yet prove $\BPP \ne \PSPACE$ because we cannot yet prove
$\P \ne \PSPACE$. The deterministic time hierarchy theorem tells us that $P \ne
\EXP$, but suprisingly, it is still only a conjecture that $\BPP \ne \NEXP$!

\subsection{Equivalent Definitions of prBPP via Amplification}

Let $\prBPP[\delta(n)]$ be the class of promise problems that are solved by a
polynomial-time Turing machine with error probability at most $\delta(n)$ on
all input strings of length $n$. Notice that $\prBPP = \prBPP[1/3]$. We have
already mentioned that for all constant $\delta_0 > 1/2$, $\prBPP[\delta_0] =
\prBPP$. However, perhaps surprisingly, the class $\prBPP[1/2]$ appears much
more powerful than $\prBPP$: for example, $\NP \subseteq \prBPP[1/2]$.%
\footnote{The non-promise class corresponding to what we call $\prBPP[1/2]$ is
called $\PP$.}

Recall the following additive Chernoff bound.

\begin{theorem}[Additive Chernoff lower tail bound]\label{thm:chernoff}
  Let $X_1, \dots, X_n$ are independent random variables where $X_i \sim
  \Bern(p)$, let $X := \sum_{i=1}^n X_i$, and let $\mu = \Exp X$. For all
  $\lambda \ge 0$, \[
    \Pr\paren*{X \le \mu - \lambda} \le \exp\paren*{-\frac{\lambda^2}{2\mu}}.
  \]
\end{theorem}

\begin{proposition}[Majority trick]\label{prop:maj}
  Let $\eps \in (0, \frac{1}{2}]$, let $p \ge \frac{1}{2} + \eps$, and let $X_1,
  \dots, X_t \sim \Bern(p)$ be i.i.d. The majority of $X_1, \dots, X_t$ is 1
  with probability at least $1 - \exp(-\eps^2 t/2)$.
\end{proposition}

\begin{proof}
  Let $X := \sum_{i=1}^t X_i$ and let $\mu = \Exp X$ and observe that $\mu \ge
  t/2 + t\eps$ and $\mu \le t$. If $X \le t/2$, then the majority is 0. By the
  Chernoff bound of Theorem~\ref{thm:chernoff}, the probability of this bad
  event is \[
    \Pr(X \le t/2) = \Pr(X \le \mu - \eps t) < \exp\paren*{-\frac{\eps^2t^2}{2\mu}} \le \exp\paren*{-\frac{\eps^2t}{2}}.
  \]
  Hence the output is 1 with probability at most $1 - \exp(-\eps^2 t/2)$.
\end{proof}

\begin{corollary}\label{cor:bpp-small-error}
  If $\delta(n)$ satisfies $\delta(n) = 2^{-\poly(n)}$ and $1/2 - \delta =
  1/\poly(n)$, then $\prBPP[\delta(n)] = \prBPP$.
\end{corollary}

\begin{proof}
  By Proposition~\ref{prop:maj}, any algorithm with success rate $\delta_0 =
  1/2 + \eps$ can be used to produce an algorithm with success rate $\delta_1 >
  \delta_0$ by repeating the algorithm independently $t :=
  2\ln(\delta_1)/\eps^2$ times and taking the majority. The conditions on
  $\delta_0$ and $\delta_1$ ensure that $t = \poly(n)$.
\end{proof}

\begin{corollary}
  $\BPP \subseteq \Ppoly$.
\end{corollary}

\begin{proof}
    Let $L \in \BPP$. By Corollary~\ref{cor:bpp-small-error}, there exists a
    probabilistic polynomial-time machine $M$ solving $L$ with error
    probability at most $2^{-2n}$.

    Fix $n$ and let $f_n : \{0, 1\}^n \to \{0, 1\}$ be the indicator for $L
    \cap \{0, 1\}^n$. Let $A_x$ be the event that $M(x, r) \ne f_n(x)$, where
    the randomness is over the choice of $r$. For any fixed $x \in \{0, 1\}^n$,
    we have
    \begin{align*}
      \Pr\brac*{A_x} \leq 2^{-2n}.
    \end{align*}
    Union bounding over all inputs of length $n$, \[
      \Pr\brac*{\bigcup_{x \in \set{0, 1}^n} A_x}
      \leq \sum_{x \in \set{0,1}^n} \Pr_r\brac*{A_x}
      \leq \sum_{x \in \set{0,1}^n} 2^{-2n}
      < 1.
    \]
    There is thus some $r^*$ such that $A_x$ does not occur for any $x$. In
    other words, $M(x,r^*) = f_n(x)$ for all inputs $x$ of length $n$. Hence the
    machine that on input $x$ and advice $r^*$ simulates $M$ on input $(x,
    r^*)$ solves $L$ in polynomial time using polynomial bits of advice.
\end{proof}


\subsection{The Circuit Acceptance Probability Problem (CAPP)}

The \emph{acceptance probability} of a circuit is the fraction of inputs for
which it outputs 1.

\begin{definition}[$\CAPP$]
  $\CAPP$ is the promise problem where
  \begin{enumerate}
    \item $\CAPP_Y$ is the set of all descriptions of circuits that have
      acceptance probability at least 2/3, and
    \item $\CAPP_N$ is the set of all descriptions of circuits that have
      acceptance probability at most 1/3.
  \end{enumerate}
\end{definition}

\begin{theorem}[$\CAPP$ is $\prBPP$-complete]\label{thm:complete}
  $\CAPP$ is complete for $\prBPP$ under deterministic polynomial-time reductions.
\end{theorem}

Recall from the reduction in Cook's theorem that every Turing machine can be
efficiently converted to an equivalent circuit for a fixed input length.
\begin{lemma}\label{lem:tm-to-circuit}
  For all $n \in \N$ there is an $O(t^2)$-time algorithm that, on input
  $\langle M \rangle$ for some Turing machine $M$, outputs a circuit $C_n$ such
  that $C_n$ accepts $x \in \{0, 1\}^n$ if and only if $x \in L(M)$.
\end{lemma}

\begin{proof}[Proof of Theorem~\ref{thm:complete}]
  The problem $\CAPP$ is easily seen to be in $\prBPP$: consider the algorithm
  that simply simulates the input circuit on a randomly chosen input. So let us
  show that every problem in $\prBPP$ can be reduced to $\CAPP$ in
  deterministic polynomial time. Let $\Pi \in \prBPP$ and let $M$ be the
  probabilistic polynomial-time Turing machine solving $\Pi$. Given an input
  $x$, by Lemma~\ref{lem:tm-to-circuit}, we can construct in polynomial-time a
  circuit $C_x$ such that $C_x(r) = M(x, r)$ for all $r$. Hence if $x \in
  \Pi_Y$, then $C_x \in \CAPP_Y$, and if $x \in \Pi_N$, then $C_x \in \CAPP_N$.
\end{proof}

\subsection{A Time-Hierarchy Theorem}

\begin{theorem}
  $\prBPTIME[T \log{T}] \subsetneq \prBPTIME[T]$
\end{theorem}

\comment{TODO: Why standard diag fails for BPP. Why it fails for prBPP. 
         Why lazy diag works for prBPP. Why it still fails for BPP.}

\subsection{Reducing Search Problems to Decision Problems}

\subsection{Two-Sided Error vs.\ One-Sided Error}

\comment{TODO: reference Lecture 5, in which the theorem below is proved}

\begin{definition}
  An \emph{$(\eps, \delta)$-sampler} is a function $s :
  \set{0,1}^{\overline{T}} \times \set{0,1}^\ell \to \set{0,1}^T$ such that for
  all $f : \set{0,1}^T \to \set{0,1}$ with probability at least $1 - \delta$ 
  over $z \in \set{0,1}^{\overline{T}}$ it holds that \[
    |\Exp_i f(s(z, i)) - \Exp_x f(x)| \le \eps.
  \]
\end{definition}

\begin{theorem}
  For all $\eps > 0$ there exists a polynomial-time computable $(\eps,
  \delta)$-sampler with $\overline{T} = \poly(T)$, $\ell = O(\log
  \overline{T}/\eps)$ and $\delta = 2^{\overline{T}^\eps - \overline{T}}$.
\end{theorem}

\begin{corollary}\label{cor:bpp-small-random}
  For all $\eps > 0$, the class $\prBPP$ can be equivalently defined with an
  error of $2^{T^\eps - T}$ instead of 1/3, where $T = T(n)$ is the number of
  random bits.

  In other words, $2^{T^\eps}$ out of the $2^T$ random bit-strings are exceptional.
\end{corollary}

\begin{theorem}\label{thm:bpp-subset-rprp}
    $\prBPP \subseteq \prRP^\prRP$.
\end{theorem}

\begin{proofsk}
    Let $\Gamma = (\Gamma_Y, \Gamma_N)$ be the following promise problem:
    \begin{itemize}
        \item $\Gamma_Y = \set{\langle C \rangle \mid \Pr\brac{C(r) = 1} = 1}$,
        \item $\Gamma_N = \set{\langle C \rangle \mid \Pr\brac{C(r) = 0} \geq 1 - 2^{\sqrt{|r|} - |r|}}$.
    \end{itemize}
    One can verify that $\Gamma$ is indeed in $\prcoRP$.

    Let $\Pi = (\Pi_Y, \Pi_N) \in \prBPP$.
    By Corollary~\ref{cor:bpp-small-random}, there is a probabilistic polynomial time 
    Turing Machine $M$ deciding $x \in \Pi_Y$ or $x \in \Pi_N$ with $2^{\sqrt{p(|x|)} 
    - p(|x|)}$ error probability over $p(|x|)$ random bits.
    
    Consider the following $\prRP^\Gamma$ algorithm that, given $x \in \Pi_Y \cup \Pi_N$, 
    decides if $x \in \Pi_Y$ or $x \in \Pi_N$.
    \begin{itemize}
        \item Compile $M(x,r)$ to a circuit $C_x(r)$ which accepts or rejects at most 
            $2^{\sqrt{|r|}}$ of its inputs (see Lemma~\ref{lem:tm-to-circuit}, 
            Theorem~\ref{thm:complete}).
        \item Fixes the first $|r|/2$ inputs of $C_x(r)$ randomly, yielding the circuit 
            $C'_x(r')$.
        \item Queries the $\Gamma$ oracle with a description of $C'_x(r')$ and returns 
            the same answer.
    \end{itemize}

    \comment{TODO: Finish up and turn into proper proof. 
    Show $\Pi_Y$ gets $1$ whp, and $\Pi_N$ gets $0$ all the time.}
\end{proofsk}

\begin{corollary}
    $\prBPP = \prP \iff \prRP = \prP$.
\end{corollary}

\begin{proof}
    If $\prBPP = \prP$, then it is clear that $\prRP = \prP$ since $\prP \subseteq
    \prRP \subseteq \prBPP$.

    Conversely, if $\prRP = \prP$, then by Theorem~\ref{thm:bpp-subset-rprp},
    $\prBPP \subseteq \prRP^\prRP = \prP^\prP = \prP$.
    Since $\prP \subseteq \prBPP$, we have $\prBPP = \prP$.
\end{proof}

\end{lecture}

  \begin{lecture}{2}{Hardness vs.\ Randomness}{January 27, 2022}
\label{lec:02}

\subsection*{The Upshot}

\begin{enumerate}
  \item Hardness and randomness are intimately connected. Suitably hard
    functions $f$ can be used to build PRGs that exponentially stretch random
    seeds into a string that appears random to circuits that cannot compute $f$
    with any significant advantage.
  \item The quintisential construction of such a PRG is the Nisan-Wigderson PRG
    (NW-PRG).
  \item The NW-PRG applies the hard function $f$ to many mostly disjoint
    projections of the seed; the projections are given by a combinatorial
    design.
  \item The NW-PRG shows that if there is constant $\varepsilon > 0$ and a
    language $L \in \DTIME[2^{O(n)}]$ that is hard for circuits of size
    $2^{\varepsilon n}$, then there is a $(1/n)$-PRG for circuits of size
    $\widetilde{O}(n)$. In particular, the existence of such a $\varepsilon$ and
    $L$ implies $\prBPP = \prP$.
  \item A key technique in the proof of the previous item is the \emph{hybrid
    argument}, which asserts that if a circuit $D$ distinguishes the uniform
    distribution from another product distribution, then it distinguishes two
    ``hybrid'' product distributions whose marginal distributions are the same
    in all but one coordinate.
\end{enumerate}

\subsection{Pseudorandomness}

\subsection{Hitting Set Generators}

\begin{definition}[Hitting Set Generator]
    A deterministic algorithm $H$ is a Hitting Set Generator (HSG) with seed length 
    $\ell(n)$ for $F = \set{f_n}_n$ if it satisfies $\Pr\brac{f(H(1^n, u_{l(n)})) = 
    1} > 0$ for all $f \in F$.
\end{definition}

\begin{theorem}
    HSGs for circuits of size $\widetilde{O}(n)$ that accept ``most'' of their inputs 
    suffice to derandomize $\prRP$.
\end{theorem}

\comment{Formalize ``most''.}

\begin{definition}[Kolmogorov Complexity]
    The Kolmogorov Complexity of $x \in \set{0, 1}^*$ is the length of the shortest 
    programs that, when run, prints $x$ and halts.
\end{definition}

\begin{theorem}
    There is an algorithm $H$ that, when given $1^n$ and oracle access to a hard 
    string $f \in \set{0, 1}^{n^2}$ with Kolmogorov Complexity $|f| - o(1)$, is 
    an HSG with seed length $\ell(n) = \log{n}$ and polynomial runtime for circuits 
    of size $\widetilde{O}(n)$ that reject at most $2^{\sqrt{n}}$ inputs.
\end{theorem}

\begin{proofsk}
    Let $f = f_1 \cdot f_2 \cdot \ldots \cdot f_n$ where $|f_s| = n$ and $\cdot$ is the 
    concatenation operator.
    Define $H(1^n, \langle s \rangle) = f_s$ where $s \in [n]$.
    We claim that $H$ is an HSG with seed length $\ell(n) = \log{n}$ and polynomial 
    runtime for circuits of size $\widetilde{O}(n)$ that reject at most $2^{\sqrt{n}}$ 
    inputs.

    Assume for a contraction that $D$ is a circuit of size $\widetilde{O}(n)$ that 
    rejects at most $2^{\sqrt{n}}$ inputs and $D(H(1^n, \langle s \rangle)) = 0$ for 
    all $s \in [n]$.
    Let $i(s)$ be the index of $f_s$ in $D^{-1}(0)$.
    The following is a short description of $f$: $D, i(1), i(2), \dots, i(n)$.
    Since $\card{D^{-1}(0)} \leq 2^{\sqrt{n}}$, we have $|i(s)| \leq \sqrt{n}$.
    The length of the description is thus $\widetilde{O}(n) + n \cdot \sqrt{n}$ which 
    contradicts our assumption that the shortest description of $f$ has length $n^2 - 
    o(1)$.

    \comment{TODO?: More formal description of the program that prints $f$}
\end{proofsk}

\subsection{Pseudorandom Generators}

\subsection{The Nisan-Wigderson Generator}

\begin{definition}[Combinatorial design]
  A collection of subsets $T_1, \dots, T_m \subseteq [d]$ is a \emph{$(\ell, a)$-design}
	\begin{enumerate}
    \item $\card{T_i} = \ell$ for all $i$, and
		\item $\card{T_i \cap T_j} < a$ for all $i \ne j$.
	\end{enumerate}
\end{definition}

We will be interested in combinatorial designs where $m = 2^\ell$, $d =
O(\ell)$, and $a = \gamma \ell$ for some constant $\gamma > 0$. The following
proposition states that, not only does such a design exist for all $\ell$ and
all $\gamma$, but it can be computed deterministically in time $O(2^\ell)$.

\begin{theorem}\label{thm:design}
  Let $\gamma > 0 $ and let $\ell, m \in \mathbb{N}$. For all $a \ge \gamma
  \log{m}$ and $d \ge e^2 \cdot 2^{1/\gamma} \cdot \ell^2/a$, there exists an
  $(\ell, a)$-design $T_1, \dots, T_m \subseteq [d]$. Moreover, such a design
  can be found deterministically in time $\poly(m, d)$.
\end{theorem}

The proof of the theorem will use the following lemma.
\begin{lemma}\label{lem:design}
  Let $a, \ell, d \in \N$ with $a \le \ell \le d$, and let $T_1, \dots, T_m \in
  \binom{[d]}{\ell}$. If $m < \binom{d}{a} / \binom{\ell}{a}^2$, then there
  exists a set $T^* \in \binom{[d]}{\ell}$ such that $\card{T^* \cap T_i} < a$
  for all $i \in [m]$. Moreover, such a set $T^*$ can be found
  deterministically in time $\poly(m, d)$.
\end{lemma}

\begin{proof}
  We first use the probabilistic method to show that such a set $T^*$ exists.
  Let $\mathbb{T}$ be a uniformly random $\ell$-set over $[d]$, let $X_j$
  be an indicator for the ``bad'' event $B_j$ that $|\mathbb{T} \cap T_j| \ge
  a$, and let $X = \sum_{j=1}^m X$ be the number of bad events. It suffices to
  show that \[
    \Exp\brac*{X} < 1,
  \]
  because then there is some realization of $\mathbb{T}$ for which all $X_j$
  are 0.

  The number of $\ell$-sets whose intersection with $T_j$ is at least $a$ is
  most $\binom{\ell}{a}\binom{d - a}{\ell - a}$: first choose $a$ elements from
  $T_j$ and then pick any $\ell - a$ elements from the remaining elements. Hence
  \[
    \Pr(B_j) \le \frac{\binom{\ell}{a}\binom{d - a}{\ell - a}}{\binom{d}{\ell}} =
    \frac{\binom{\ell}{a}^2}{\binom{d}{a}},
  \]
  where we used the identity $\binom{d}{\ell}\binom{\ell}{a} =
  \binom{d}{a}\binom{d - a}{\ell - a}$ in the last step. If $m < \binom{d}{a} /
  \binom{\ell}{a}^2$, then \[
    \Exp\brac*{X} =
    \sum_{j=1}^m \Pr(B_j) \le m \cdot \frac{\binom{\ell}{a}^2}{\binom{d}{a}} < 1,
  \]
  as desired.

  Now we describe how to construct such a set $T^*$ deterministically using the
  \emph{method of conditional probabilities}. At a high level, the algorithm
  simply iterates over the elements $i \in [d]$ and in each iteration decides
  whether to include $i$ in $T^*$ based on a conditional expectation
  calculation. For example, suppose we want to decide if the element $1$ should
  be included. We have \[
    1 > \Exp\brac*{X} = \Pr(1 \in \mathbb{T}) \cdot \Exp\brac*{X \mid 1 \in \mathbb{T}} 
      + \Pr(1 \not\in \mathbb{T}) \cdot \Exp\brac*{X \mid 1 \not\in \mathbb{T}},
  \]
  and so one of $\Exp\brac{X \mid 1 \in \mathbb{T}}$ or $\Exp\brac{X \mid 1 \not\in
  \mathbb{T}}$ is less than one. If $\Exp\brac{X \mid 1 \in \mathbb{T}} < 1$,
  then it is safe to add $1$ to $T^*$. Inductively, when we consider adding
  element $i$ to $\mathbb{T}$, one of the two expectations will be less than
  one, which gives us a decision for $i$.

  The only thing left to show is that for any set $T \subseteq [i]$, we can
  compute $\Pr(B_j \mid \mathbb{T} \cap [i] = T)$ exactly. This is a bit of an
  ugly calculation, but only involves basic combinatorics; we leave the details
  to the reader.
\end{proof}
%    \Pr(B_j \mid \mathbb{T} \cap [i] = T) = \binom{d - i}{\ell - |T|}^{-1}
%    \sum_{b=a}^{\ell} \binom{|S_j \cap \set{i+1, \dots, d}|}{b-|T \cap S_j|}\binom{d - \ell}{\ell}

Now we return to the proof of Theorem~\ref{thm:design}.

\begin{proof}[Proof of Theorem~\ref{thm:design} using Lemma~\ref{lem:design}]
  A routine calculation shows that for $a \ge \gamma \log{m}$ and $d \ge e^2
  \cdot 2^{1/\gamma} \cdot \ell^2/a$, we have $m \le \binom{d}{a} /
  \binom{\ell}{a}^2$. Indeed,
  \begin{align*}
    \binom{d}{a} \cdot \binom{\ell}{a}^{-2}
    &\ge \paren*{\frac{d}{a}}^a \cdot \paren*{\frac{e \ell}{a}}^{-2a}\\
    &\ge \paren*{\frac{e^2 \cdot 2^{1/\gamma} \cdot \ell^2}{a^2}}^{a} \cdot \paren*{\frac{e^2 \ell^2}{a^2}}^{-a}\\
    &= 2^{a / \gamma}\\
    &\ge m.
  \end{align*}
  Hence we can iteratively apply Lemma~\ref{lem:design} $m$ times to obtain the
  sets $T_1, \dots T_m$.
\end{proof}

\end{lecture}

  \input{lec03.tex}
  \newcommand{\fws}{\ensuremath{ f^{\text{ws}} }}

\begin{lecture}{4}{Uniform Hardness vs.\ Randomness}{February 10, 2022}
\label{lec:04}

\subsection*{The Upshot}

\begin{enumerate}
  \item Uniform distinguishers for the NW-PRG with hard function $f$ can be
    used to learn $f$ with $1/\poly(n)$ advantage with high probability.
  \item If the hard function $f$ has sufficient structure, the $1/\poly(n)$
    advantage can be used to bootstrap a reconstruction of $f$.
  \item LLDCs and downward self-reducibility play a crucial role in the
    bootstrapping algorithm.
\end{enumerate}


\subsection{Uniform Circuits}

So far we have been studying PRGs for non-uniform circuits. As a consequence,
our results so far require problems in $\E$ that are hard for non-uniform models.
In this lecture we investigate the analgous questions for uniform models of
computation.


\begin{definition}[Uniform circuits]
	Uniform circuits are circuits that can be printed by a Turing Machine i.e.\ when the input to the TM is $1^n$ the output is circuit $C_n$.
\end{definition}

\begin{definition}[$\eps$-PRG for uniform circuits]
  An algorithm $G$ is an \emph{$\eps$-PRG for uniform circuits generated in
  time $t(n)$} if for every Turing machine $D$ with running time $t(n)$ and
  large enough $n \in \N$
	\[
		\Pr\brac{D(1^n) \text{ is an $\eps$-distinguisher for } G(1^n, u_{l(n)})} \leq \eps.
	\]
\end{definition}

\begin{definition}\label{p-sample-dist}
  Let $\overline{x} = \set{\overline{x}_n}$ be a sequence of distributions over
  $\set{0, 1}^n$ and let $p : \N \to \N$. We say that $\overline{x}$ is
  \emph{$p$-time samplable} if there is a probabilistic machine $S$ with
  runtime $p(n)$ such that $S(1^n)$ has the distribution $\overline{x}(n)$.
\end{definition}

\begin{theorem}\label{thm:DR-on-average}
  Assume that for all polynomials $p(n)$ there is an $\eps$-PRG $G$ for uniform
  circuits generated in time $p(n)$ that stretches a $\log(n)$-sized seed. Then
  for all $L \in \BPP$ and poly-time samplable distributions $\overline{x}$,%
  \footnote{Think of $\overline{x}$ as an adversary (limited to polynomial
  time) trying to generate hard inputs with $\geq \eps$ probability.} there
  exists a language $L' \in \textsf{P}$ such that
	\[
		\Pr_{x \sim \overline{x}}\brac{L(x) \neq L'(x)} \leq \eps.
	\]
\end{theorem}

\begin{proof}
	Let $S$ be the sampling algorithm for $\overline{x}$, and let $M = M_L$
	be the $\BPP$ machine for $L$. We define $L'$ as: given $x$, enumerate
	over seeds $s$ of $G$, and output
	\[ \textsf{MAJ}_{s} \set{M(x, G(s)) }. \]
	Since the seed length $s$ is $O(\log n)$, the algorithm above runs in
	polynomial time, and hence $L' \in \P$.
	Further, for each $x$, we define:
	\[
		\Delta(x) = \card{
			\Pr_{r \sim \overline{u}_n}         \brac{ M(x, r) = 1}
                  - \Pr_{s \sim \overline{u}_{\ell(n)}} \brac{ M(x, G(s)) = 1}
		  }.
	\]

	We suppose (towards a contradiction) that $L'$ differs from $L$ on more
	than an $\eps$-fraction of inputs. Since $M$ is a $\BPP$ machine for $L$
	this means that
	$\Pr_{x \sim \overline{x}_n}\brac{\Delta(x) > \eps} > \eps$. Then consider
	the algorithm $D$: on the input $1^n$, sample $x \sim \overline{x}_n$ using
	$S(1^n)$, and then output the circuit $D_x(r) = M(x, r)$.
	Then
  \begin{align*}
		\Pr\brac{D \text{ prints a circuit that } \eps \text{-distinguishes } G(1^n, u_{\ell})}
    &= \Pr\brac{\Delta(S(1^n)) > \eps}\\
    &= \Pr_{x \sim \overline{x}_n}\brac{\Delta(x) > \eps}\\
    &> \eps,
  \end{align*}
	which contradicts the fact that $G$ is an $\eps$-PRG.
\end{proof}

\comment{This bit needs some exposition between the theorems / conjecture.}

\begin{theorem}
	If
	$\SPACE\brac{O(n)}$ is hard for $\BPTIME\brac{2^{\eps n}}$
	then
	$\RP = \P$ ``on average''.
\end{theorem}


\begin{theorem}
	There exists a language $\fws \in \P^{\sharpp}$ such that if $\fws$
	is hard for $\BPTIME\brac{2^{\eps n}}$ then
	$\BPP = \P$ ``on average''.
\end{theorem}

The direct analogue of the hardness to derandomization result for non-uniform
circuits from Lecture~\ref{lec:03} is the following conjecture:

\begin{conjecture}
	There exists an $\eps > 0$, such that $\E$ is hard for
	$\BPTIME\brac{2^{\eps n}}$ if and only if for all polynomials $p(n)$ there
	is a $(1 / n)$-PRG for uniform circuits generated in time $p(n)$ with
	runtime $\poly(n)$ that stretches a $\log(n)$ sized seed.
\end{conjecture}

\subsection{Learning via Uniform Distinguishers}

\comment{Restate / refer to the Nisan-Wigderson PRG here.}

\begin{definition}
  Let $f_n : \set{0, 1}^n \to \set{0, 1}.$ A probabilistic algorithm $A$ with
  oracle access to $f_n$ \emph{learns $f_n$ with success $\delta$ and advantage
  $\eps$} if on input $1^n$, $A$ produces a circuit $C_n$ such that
  $\Pr_{x \sim \{0, 1\}^n} \brac{C_n(x) = f_n(x)} \geq 1/2 + \eps$ with
  probability $1 - \delta$.
\end{definition}

\begin{proposition}\label{prop:learning}
  Let $\ell = \log{n}$, let $\gamma \in (0, 1/2)$, and let $T_1, \dots, T_n$ be
  a $(\ell, \gamma \ell)$-design over $[O(\ell)]$. Finally, let $f :
  \set{0,1}^\ell \to \set{0, 1}$ and let $G^f$ be the Nisan-Wigderson PRG with
  oracle $f$ and design $T_1, \dots, T_n$. If $D$ is a uniform circuit on $n$
  inputs generated in time $\poly(n)$ that is not $(1/n)$-fooled by $G^f$, then
  there is a $\poly(n)$-time algorithm to learn $f$ with advantage $1/n^2$ and
  success $1/n$.
\end{proposition}

\begin{proof}
  By definition, if $D$ is not $(1/n)$-fooled by $G^f$, then
  \[
    \card{\Pr\brac{D(G(1^n, s)) = 1} - \Pr\brac{D(\overline{u}_n) = 1}} > 1/n.
  \]
  \comment{Should the input to $G$ really be $1^n$ above?}

  As in the hybrid argument (\comment{ref. lecture 2 here}), we define the
  distributions
  \begin{align*}
    \overline{H}_0 &= (f(x_1), \ldots, f(x_n)),\\
    \overline{H}_i &= (\overline{u}_i, f(x_{i + 1}), \ldots, f(x_n)),\\
    \overline{H}_n &= \overline{u}_n,
  \end{align*}
  where $x_i$ is the projection of the seed given to the PRG onto the
  coordinates $T_i$. Recall the (easy) lemma that there is an $i$ such that:
  \[
    \card{\Pr\brac{D(\overline{H}_{i - 1}) = 1} - \Pr\brac{D(\overline{H}_i) = 1}} > 1 / n^2.
  \]
  Roughly speaking, the learning algorithm will guess the value of this $i$ and
  exploit the fact that $D$ can differentiate the bit $f(x_i)$ from a uniformly
  random bit. More concretely, we describe the algorithm $A$:
  \begin{enumerate}
    \item Construct the a $(\ell, \gamma \ell)$-design $T_1, \dots, T_n$ in
      time $\poly(n)$ using Theorem~\ref{thm:design} from
      Lecture~\ref{lec:02}.
    \item Choose a random $i \in \brac{n}$.
    \item Choose a random $z^* \in \set{0, 1}^{\ell(n) - \ell}$. Here $s$
      (the seed of the PRG) projected to $T_i$ will be equal to $x_i$,
      and the remaining coordinates will be filled by $z^*$.
    \item Choose a random bit $\sigma$.
    \item Query the oracle for $f$ on all possible inputs
      $x_{i + 1}, \ldots , x_n$. Note that at most $\ell / 100$ bits of
      each of these $x_j$'s are unknown after fixing $z^*$, so there
      are at most $n \cdot 2^{\ell / 100}$ queries to the oracle.
    \item Choose a random $u^* \in \set{0, 1}^{i - 1}$.
    \item Output the circuit that takes input $x_i$ and simulates
      $D$ on $(u^*, \sigma, f(x_{i + 1}), \ldots , f(x_n))$ and outputs
      $\lnot \sigma$ if $D$ accepts, and $\sigma$ otherwise.
  \end{enumerate}
  To complete the proof, we claim:

  \begin{claim}
    With probability $1 / n$, $\Pr\brac{C_n(x) = f_n(x)} = 1/2 + 1/n^2$.
  \end{claim}
  To see the claim, suppose without loss of generality that the probability
  that $D$ accepts on an input sampled from $\overline{H}_{i}$ is larger
  than the probability it accepts on one from $\overline{H}_{i - 1}$.
  For brevity, define the distribution $\overline{H}'_{i - 1}$ which differs
  from $\overline{H}_{i - 1}$ only in the $i$-th coordinate, where it is
  $\lnot f(x_i)$.
  Then note that since the distributions $\overline{H}_i$ and
  $\overline{H}_{i - 1}$ differ only in their $i$-th bit, and this bit is
  either $f(x_i)$ or $\lnot f(x_i)$ with equal probability in
  $\overline{H}_i$:
  \[
    \Pr\brac{D(\overline{H}_{i}) = 1}
    = \frac 12 \Pr\brac{D(\overline{H}_{i - 1}) = 1}
    + \frac 12 \Pr\brac{D(\overline{H}'_{i - 1}) = 1}.
  \]
  But this means that $\Pr\brac{D(\overline{H}'_{i - 1}) = 1} >
  \Pr\brac{D(\overline{H}_{i - 1}) = 1} + 2/n^2$.
  Now, $C_n$ outputs $f(x_i)$ correctly if either:
  \begin{itemize}
    \item $\sigma = f(x_i)$, and $D$ rejects, which happens with
      probability $1/2 \cdot \Pr\brac{D(\overline{H}_{i - 1}) = 0}$.
    \item $\sigma \neq f(x_i)$ and $D$ accepts, which happens with
      probability $1/2 \cdot \Pr\brac{D(\overline{H}'_{i - 1}) = 1}$.
  \end{itemize}
  And hence the probability that $C_n(x) = f_n(x)$ is
  \[
    \frac 12 \paren*{
      \Pr\brac{D(\overline{H}_{i - 1}) = 0} +
      \Pr\brac{D(\overline{H}'_{i - 1}) = 1}
      }
      > 1/2 + 1/n^2.
    \]
\end{proof}

\begin{corollary}
  \label{corr:learning-boost-success}
  If there is a $(1/n)$-distinguisher for the NW-PRG $G^f$, then there is an
  algorithm that learns $f$ with success $1 - 1/\delta$ and advantage
  $1/(2n^2)$ that runs in time $\poly(n, \log(1/\delta))$.
\end{corollary}

\begin{proof}
  Let $A$ be the learning algorithm for $f$ with success $1/n$ and advantage
  $1/n^2$ promised by Proposition \ref{prop:learning}. By running $A$
  independently $n \log(1/\delta)$ times, the probability that none of the
  circuits produced has advantage $1/n^2$ is $1 - (1 - 1/n)^{n\log(1/\delta)} >
  1 - \delta$. We can estimate the acceptance probability of each of the
  $n\log(1/\delta)$ circuits to within an additive error of $1/(2n^2)$ with
  probability $1 - \delta$ by running it on $O(n^4 \log(1/\delta))$ random
  inputs. Hence if such a circuit exists, we will find it in time $\poly(n,
  1/\delta)$. Reparameterizing $\delta$ gives the result.
\end{proof}

\subsection{Bootstrapping via Downward Self-Reducibility and LLDCs}

\begin{definition}
  $f$ is \emph{downward self-reducable (DSR)} if there exists an algorithm $A$
  such that when $A$ gets $x \in \set{0, 1}^n$ and oracle access to $f$ on $n -
  1$ bits, it outputs $f(x)$ in linear time.%
  \footnote{Linear time is not critical, it can be replaced (for example) by
	quadratic time ... but not $\poly$-time (???).}
\end{definition}

For example, $\SAT$ and $\PERM$ are DSR.

\begin{proposition}
  There exists a function $\fws \in \P^{\sharpp}$ such that:
  \begin{enumerate}
    \item $\fws$ is DSR.
    \item For all $\ell \in \N$ the truth table of $\fws$ is a codeword
      which is:
      \begin{enumerate}
        \item Locally List Decodable from agreement $\eps(\ell) = 2^{-\ell /
          100}$ in time $\poly(1/\eps)$ with list size $\poly(1 / \eps)$.
        \item Uniquely Decodable from agreement $0.99$ in time $\poly(n)$.
      \end{enumerate}
  \end{enumerate}
\end{proposition}

\begin{theorem}\label{thm:uniform-hard-random-sharpP}
  If $\fws$ is hard for $\BPTIME\brac{2^{\eps n}}$ then for all polynomials
  $p(n)$ there is a $(1 / n)$-PRG for uniform circuits generated in time $p(n)$
  with $\log$-sized seed and polynomial running time.
\end{theorem}

\begin{proofsk}
  The proof is by reconstruction: Assume there is some uniform distinguisher
  $D$ that is not fooled by the PRG. We will use $D$ to give a
  $\BPTIME\brac{2^{\eps n}}$ algorithm for $\fws$.

  Let $x \in \{0, 1\}^n$; we want to compute $\fws(x)$ efficiently. The
  algorithm iteratively constructs circuits $F_i$ that computes $\fws$ on all
  inputs of length $i$. We can compute $F_1$ in constant time. Now describe how
  to compute $F_i$ from $F_{i-1}$:
  \begin{enumerate}
    \item Assume that $F_{i - 1}$ computes $\fws_{i - 1}$.
    \item Let $A$ be the learning algorithm for
      Corollary~\ref{corr:learning-boost-success} with $\delta = 1/n^2$. Run
      $A$ on input $1^i$, answering queries $\fws_i(x)$ using the downward
      self-reducability of $\fws$ and the circuit $F_{i - 1}$.
    \item With high probability, $A$ returns a circuit that agrees with
      $\fws_i$ on a $1/2 + 1/i^2$ fraction of the inputs.
    \item Compute a list of circuits via the LLDC decoder with the previous
      circuit as the corrupt codeword.
    \item Using random sampling and the downward self-reducability of $\fws$
      and $F_{i - 1}$, find a circuit that agrees with $\fws_i$ on a $0.99$
      fraction of the inputs.
    \item Combine this with a local unique decoder.
  \end{enumerate}
  To tie this up, observe that there are $\ell$ steps of the loop above, and in
  each step we run in time $\poly(2^{\ell / c})$ and output a circuit of size
  $\poly(2^{\ell / c})$.
\end{proofsk}

\end{lecture}


\end{document}
