\chapter{Hardness Amplification via Error-Correcting Codes}
\label{lec:03}

In \Cref{lec:02}, we saw that if there is a language $L$ in $\E{}$ that cannot
be computed on, say, 51\% of inputs by a circuit of size $2^{\eps n}$, then $\BPP = \P$. In
this chapter, we will show that we can
make a signficantly weaker assumption and obtain the same conclusion: If there
is a language in $\E$ that cannot be computed on 100\% of inputs by a circuit of size $2^{\eps
n}$, then $\BPP = \P$.

The main technique is called \emph{hardness amplification}: Given a problem that is hard to compute on 51\% of inputs, we construct a new problem that is hard to compute on 100\% of inputs.

\section{A Primer on Coding Theory}

\section{Hardness Amplification via Locally List Decodable Codes}

\section{Reed-Solomon Codes}

\section{Reed-Muller Codes}
